\documentclass{ximera}
%% You can put user macros here
%% However, you cannot make new environments

\usepackage[letterpaper, total={6in, 8in}]{geometry}

\usepackage{tikz}
\usepackage{tkz-euclide}
\usetkzobj{all}

\tikzstyle geometryDiagrams=[ultra thick,color=blue!50!black]

%\usepackage{enumerate}
\usepackage{euler}

 
\usetikzlibrary{shapes,snakes}
\tikzset{
  dot hidden/.style={},
  line hidden/.style={},
  dot colour/.style={dot hidden/.append style={color=#1}},
  dot colour/.default=black,
  line colour/.style={line hidden/.append style={color=#1}},
  line colour/.default=black
}

%\begin{sagesilent}
def mydata(k):
    import numpy as np
    # Random test data
    np.random.seed(123)
    all_data = [np.random.normal(0, std, 100) for std in range(1, k)]
    return all_data

#boxplot
def seabox(my_data,mystr):
    mydata.sort()
    import seaborn as sns
    sns.set_style("whitegrid")
    #tips = sns.load_dataset("tips")
    ax = sns.boxplot(x=my_data)
    ax.set(xlim=(0, len(mydata)+10))    
#    ax.get_figure().savefig(mystr,bbox_inches='tight')
    
    
def seahist(data_list,bin,mystr):
    import matplotlib.pyplot as plt    
    import seaborn as sns
    ax=sns.distplot(data_list, color='red',bins=bin, kde=False)
    ax.set(xlim=(0, None))    
    ax.get_figure().savefig(mystr,bbox_inches='tight')
    plt.savefig(mystr)
    plt.clf()
    
    #histogram
def myhist(k,b,m,M):
    from sage.plot.histogram import Histogram
    #k=data size
    # b= number of bins
    # random datA range
    a=histogram([randint(m,M) for _ in range(k)], bins=b)
    return a

#barchart & table
'''data1=[
['Zoo','giraffes', 'orangutans', 'monkeys'], ['SF Zoo', 20, 14, 23],['LA Zoo',12, 18, 29]
]'''
def mytb(data1,table_name,bar_name):
    import plotly.plotly as py
    import plotly.graph_objs as go
    from plotly.tools import FigureFactory as FF 
    from IPython.display import Image

    table1=FF.create_table(data1)
    py.iplot(table1,filename=table_name[:-4])
    py.image.save_as(table1, filename=table_name)
    Image(table_name)
    width=len(data1[0])
    height=len(data1)
    show(width)
    trace=[]
    d0=data1[0][1:width]
    show(d0)
    for i in range(height-1):
        show(i)
        d1=data1[i+1][1:width]
        show(d1)
        trace.append(
            go.Bar(
            x=d0,
            y=d1,
            name=data1[i+1][0]
                   )
                     )
    data = trace
    layout = go.Layout(
        barmode='group'
    )

    fig = go.Figure(data=data, layout=layout)
    py.iplot(fig, filename=bar_name[:-4])
    py.image.save_as(fig, filename=bar_name)
    Image(bar_name)
#piechart :
def mypie(labels,values,pie_name):
    import plotly.plotly as py
    import plotly.graph_objs as go
    from IPython.display import Image
    trace=go.Pie(labels=labels,values=values)
    py.iplot([trace], filename=data_name[:-4])
    py.image.save_as([trace],filename=pie_name)
    Image(pie_name)
    
\end{sagesilent}



\title{Significance and Confidence Intervals - HW7.5}
\author{Oguz Kurt}

\begin{document}
\begin{abstract}
\empty
\end{abstract}
\maketitle
\begin{sagesilent}
def c(n,k):
    if 0<=k and k <=n:
        return binomial(n,k)
    else:
        return "error"

def p(n,k):
    if 0<=k and k <=n:
        return binomial(n,k)*factorial(k)
    else:
        return "error"
def m(n,mylist):
    l=len(mylist)
    return factorial(n)/(prod([factorial(mylist[i]) for i in range(l)]))
\end{sagesilent}

%\geogebra{sSsy52bG}


\section*{Syntax}
Here is a reminder of the way to input some numbers. Please, use the parsing power of your browser instead of using a calculator.

\hspace{0.5cm}


\begin{tabular}{c|c}
Operation & Syntax  
\\
\hline
$a+b$ & {\color{red} \verb!a+b!} 
\\
\hline
$a- b$ & {\color{red} \verb!a-b!} 
\\
\hline
$a\cdot b$ & {\color{red} \verb!a*b! } 
\\
\hline
$a/b=\frac{a}{b}$ & {\color{red} \verb!a/b!}
\\
\hline
$a^b$ & {\color{red} \verb|a**b| or \verb|a^b|} 
\\
\hline
$\sqrt{a}$ & {\color{red} \verb|sqrt(a)| or \verb|a**(1/2)|} 
\\
\hline
$a!$ & {\color{red} \verb|a!|} 
\\
\hline
$\binom{a}{b}$ & {\color{red} \verb|a!/(b!*(a-b)!)|}
\\
\hline
$P_k^n$ & {\color{red} \verb|a!/(a-b)!|}
\\
\hline
$e^x$ & {\color{red} \verb|exp(x)|}
\\
\end{tabular}

\hspace{1cm}
\begin{definition}[Significance Levels] Suppose  $Z$ has the standard normal distribution (or Gaussian distribution) and $0<\alpha<1$.
\begin{itemize}
    \item If we focus on  $\Pr(Z\geq z_{\alpha})=\alpha$, then  we study the {\bf RIGHT tail} of the given normal curve. Here, $z_{\alpha}$ is almost always positive as our main focus is very small values of $\alpha$. In this case, $\Pr(Z\geq z)\geq\alpha$ for any $z<z_{\alpha}$. Hence, we call $z$ is {\bf statistically significant} if $z\leq z_{\alpha}$. We also call $(0,z_{\alpha}]$ as the {\bf critical region} for the given right-tailed problem. 
    \item If we focus on  $\Pr(X\leq z_{\alpha})=\alpha$, then  we study the {\bf LEFT tail} of the given normal curve. Here, $z_{\alpha}$ is almost always negative  as our main focus is very small values of $\alpha$. In this case, $\Pr(Z\leq z)\geq \alpha$ for any $z\geq z_{\alpha}$. Hence, we call $z$ is {\bf statistically significant} if $z\geq z_{\alpha}$. We also call $[z_{\alpha},0$ as the {\bf critical region} for the given left-tailed problem. 

    \item If we focus on  $\Pr(|X|\geq z_{\alpha/2})=2\Pr(X\geq z_{\alpha/2})=2(\alpha/2)=\alpha$, then  we study the {\bf two tails} of the given normal curve, both. Here, $z_{\alpha/2}>0$ but we focus on both tails. So, both $\pm z_{\alpha/2}$ matter. In this case, $\Pr(|Z|\geq z)\geq \alpha$ whenever $-z_{\alpha/2}\leq z \leq z_{\alpha/2}$. Hence, we call $z$ is {\bf statistically significant} if $-z_{\alpha/2}\leq z \leq z_{\alpha/2}$. We also call $[-z_{\alpha/2},z_{\alpha/2}]$ as the {\bf critical region} for the given left-tailed problem. 
\end{itemize}

In each of the above case, $\alpha$  is called the {\bf significance level}. In this case, $1-\alpha$ is called the {\bf confidence level}.

If $X\sim \mathcal N(\mu,\sigma)$, we can always convert our critical interval for z-scores to the critical interval for $X$ using $\frac{X-\mu}{\sigma}=Z$. In each case, our critical interval is also called the {\bf confidence interval}.
\end{definition}

\begin{remark}
Being in the confidence interval and having the respective probability greater than or equal to $\alpha$ are equivalent. 
\end{remark}
\begin{theorem}
Let $X$ have the normal distribution with mean $\mu$ and standard deviation $\sigma$; that is, $X\sim \mathcal N(\mu,\sigma)$. Then the z-scores of $X$ has the standard normal distribution, that is, $$Z=\frac{X-\mu}{\sigma}\sim \mathcal N(0,1).$$ 
\end{theorem}

\begin{problem}
Suppose $Z\sim \mathcal N(0,1)$. Calculate the interval for $90\%$ confidence if we only focus on the right tail. 
\begin{explanation}
$\alpha=1-0.90=0.10$.
$\Pr(Z\geq z_{0.10})=0.10$ is satisfied for $z_{0.10}=\answer[tolerance=0.001]{1.285}$. Hence, our confidence interval for $z$ is \wordChoice{\choice[correct]{less than or equal to} \choice{greater than equal to}} $\answer[tolerance=0.001]{1.285}$.

\end{explanation}
\end{problem}

\begin{problem}
Suppose $X\sim \mathcal N(100,10)$. Calculate the interval for $90\%$ confidence if we only focus on the right tail. 
\begin{explanation}
$\alpha=1-0.90=0.10$.
$\Pr(Z\geq z_{0.10})=0.10$ is satisfied for $z_{0.10}=\answer[tolerance=0.001]{1.285}$. Hence, our confidence interval for X is \wordChoice{\choice[correct]{less than or equal to} \choice{greater than or equal to}} $\answer[tolerance=0.01]{\sage{100+1.285*10}}$.

\end{explanation}
\end{problem}


\begin{problem}
Suppose $Z\sim \mathcal N(0,1)$. Calculate the interval for $95\%$ confidence if we only focus on the left tail. 
\begin{explanation}
$\alpha=1-0.95=0.05$.
$\Pr(Z\leq z_{0.10})=0.05$ is satisfied for $z_{0.10}=\answer[tolerance=0.001]{-1.645}$. Hence, our confidence interval for $z$ is \wordChoice{\choice[correct]{greater than or equal to} \choice{less than or equal to}} $\answer[tolerance=0.001]{-1.645}$.

\end{explanation}
\end{problem}


\begin{problem}
Suppose $X\sim \mathcal N(100,10)$. Calculate the interval for $95\%$ confidence if we only focus on the left tail. 
\begin{explanation}
$\alpha=1-0.95=0.05$.
$\Pr(Z\leq z_{0.05})=0.05$ is satisfied for $z_{0.05}=\answer[tolerance=0.001]{-1.645}$. Hence, our confidence interval for $X$ is \wordChoice{\choice[correct]{greater than or equal to} \choice{less than or equal to}} $\answer[tolerance=0.001]{\sage{100-1.645*10}}$.

\end{explanation}
\end{problem}

\begin{problem}
Suppose $Z\sim \mathcal N(0,1)$. Calculate the interval for $80\%$ confidence if we focus on both tails. 
\begin{explanation}
$\alpha=1-0.80=0.20$.
$\Pr(|Z|\geq z_{0.10})=0.20$ is satisfied for $z_{0.10}=\answer[tolerance=0.001]{1.285}$. Hence, our confidence interval is $[\answer{-1.285},\answer{1.285}]$. 

\end{explanation}
\end{problem}


\begin{problem}
Suppose $X\sim \mathcal N(100,10)$. Calculate the interval for $80\%$ confidence if we focus both tails. 
\begin{explanation}
$\alpha=1-0.80=0.20$.
$\Pr(|Z|\geq z_{0.10})=0.20$ is satisfied for $z_{0.10}=\answer[tolerance=0.001]{1.285}$. Hence, our confidence interval  for $X$ is
$$[\answer{\sage{100-1.285*10}},\answer{\sage{100+1.285*10}}].$$

\end{explanation}
\end{problem}

\end{document}
