\documentclass{ximera}
%% You can put user macros here
%% However, you cannot make new environments

\usepackage[letterpaper, total={6in, 8in}]{geometry}

\usepackage{tikz}
\usepackage{tkz-euclide}
\usetkzobj{all}

\tikzstyle geometryDiagrams=[ultra thick,color=blue!50!black]

%\usepackage{enumerate}
\usepackage{euler}

 
\usetikzlibrary{shapes,snakes}
\tikzset{
  dot hidden/.style={},
  line hidden/.style={},
  dot colour/.style={dot hidden/.append style={color=#1}},
  dot colour/.default=black,
  line colour/.style={line hidden/.append style={color=#1}},
  line colour/.default=black
}

%\begin{sagesilent}
def mydata(k):
    import numpy as np
    # Random test data
    np.random.seed(123)
    all_data = [np.random.normal(0, std, 100) for std in range(1, k)]
    return all_data

#boxplot
def seabox(my_data,mystr):
    import seaborn as sns
    sns.set_style("whitegrid")
    #tips = sns.load_dataset("tips")
    ax = sns.boxplot(x=my_data)
    ax.set(xlim=(0, None))    
    ax.get_figure().savefig(mystr,bbox_inches='tight')

#histogram
def myhist(k,b,m,M):
    from sage.plot.histogram import Histogram
    #k=data size
    # b= number of bins
    # random datA range
    a=histogram([randint(m,M) for _ in range(k)], bins=b)
    return a

#barchart & table
'''data1=[
['Zoo','giraffes', 'orangutans', 'monkeys'], ['SF Zoo', 20, 14, 23],['LA Zoo',12, 18, 29]
]'''
def mytb(data1,table_name,bar_name):
    import plotly.plotly as py
    import plotly.graph_objs as go
    from plotly.tools import FigureFactory as FF 
    from IPython.display import Image

    table1=FF.create_table(data1)
    py.iplot(table1,filename=table_name[:-4])
    py.image.save_as(table1, filename=table_name)
    Image(table_name)
    width=len(data1[0])
    height=len(data1)
    show(width)
    trace=[]
    d0=data1[0][1:width]
    show(d0)
    for i in range(height-1):
        show(i)
        d1=data1[i+1][1:width]
        show(d1)
        trace.append(
            go.Bar(
            x=d0,
            y=d1,
            name=data1[i+1][0]
                   )
                     )
    data = trace
    layout = go.Layout(
        barmode='group'
    )

    fig = go.Figure(data=data, layout=layout)
    py.iplot(fig, filename=bar_name[:-4])
    py.image.save_as(fig, filename=bar_name)
    Image(bar_name)
#piechart :
def mypie(labels,values,pie_name):
    import plotly.plotly as py
    import plotly.graph_objs as go
    from IPython.display import Image
    trace=go.Pie(labels=labels,values=values)
    py.iplot([trace], filename=data_name[:-4])
    py.image.save_as([trace],filename=pie_name)
    Image(pie_name)
    
\end{sagesilent}



\title{Chebyshev's Inequality - HW5}
\author{Oguz Kurt}

\begin{document}
\begin{abstract}
\empty
\end{abstract}
\maketitle

\begin{sagesilent}
def c(n,k):
    if 0<=k and k <=n:
        return binomial(n,k)
    else:
        return "error"

def p(n,k):
    if 0<=k and k <=n:
        return binomial(n,k)*factorial(k)
    else:
        return "error"
def m(n,mylist):
    l=len(mylist)
    return factorial(n)/(prod([factorial(mylist[i]) for i in range(l)]))
\end{sagesilent}

\section*{Syntax}
Here is a reminder of the way to input some numbers. Please, use the parsing power of your browser instead of using a calculator.

\hspace{0.5cm}


\begin{tabular}{c|c}
Operation & Syntax  
\\
\hline
$a+b$ & {\color{red} \verb!a+b!} 
\\
\hline
$a- b$ & {\color{red} \verb!a-b!} 
\\
\hline
$a\cdot b$ & {\color{red} \verb!a*b! } 
\\
\hline
$a/b=\frac{a}{b}$ & {\color{red} \verb!a/b!}
\\
\hline
$a^b$ & {\color{red} \verb|a**b| or \verb|a^b|} 
\\
\hline
$\sqrt{a}$ & {\color{red} \verb|sqrt(a)| or \verb|a**(1/2)|} 
\\
\hline
$a!$ & {\color{red} \verb|a!|} 
\\
\hline
$\binom{a}{b}$ & {\color{red} \verb|a!/(b!*(a-b)!)|}
\\
\hline
$P_k^n$ & {\color{red} \verb|a!/(a-b)!|}
\\
\hline
$e^x$ & {\color{red} \verb|exp(x)|}
\\
\end{tabular}

\hspace{1cm}

\begin{problem}
A class of second graders has mean height of five feet with standard deviation of one inch. At least what percent of the class must be between 4'10''and 5'2''? 
\begin{hint}
\item $1'=12''$
\item Use Chebyshev's Inequality.
\end{hint}
\begin{prompt}%err
Let $X$ be the random variable recording the length of students. $$\Pr(4'10''\leq X \leq 5'2'')\geq\answer{0.75}=\answer{75}\%$$
\end{prompt}
\end{problem}


\begin{problem}
Computers from a particular company are found to last on average for three years without any hardware malfunction, with standard deviation of two months. At least what percent of the computers last between 31 months and 41 months?
\begin{hint}
1 year is 12 months.
\end{hint}
\begin{prompt}
Let $X$ be the time it takes  for a computer from this company to work without any mulfunction.
$$\Pr(31mo\leq X \leq 41mo)\geq\answer{\sage{1 - 1/((2.5)^2)}}=\answer{84}\%$$

\end{prompt}
\end{problem}



\begin{problem}
Bacteria in a culture live for an average time of three hours with standard deviation of 10 minutes. At least what fraction of the bacteria live between two and four hours?
\begin{prompt}
Let $X$ be the life time of bacteria culture in minutes.
$$\Pr(2<X<4)\geq \answer{ \sage{1-1/6^2}}=\answer[tolerance=0.05]{ \sage{100-100/6^2}}\%$$
\end{prompt}
\end{problem}


\begin{problem}
What is the smallest number of standard deviations from the mean that we must go if we want to ensure that we have at least 50\% of the data of a distribution?
\begin{prompt}
Let's say we have a random variable $X$ with average $\mu$ and standard deviation $\sigma$. 

$$\Pr\left(\mu-\answer{\sqrt{2}}\cdot \sigma< X < \mu-\answer{\sqrt{2}}\cdot \sigma \right)=\Pr\left(\left|\frac{X-\mu}{\sigma}\right|<\answer{\sqrt{2}}\right)\geq 0.5 $$
\end{prompt}
\end{problem}

\begin{problem}

Bus route \#25 takes a mean time of 50 minutes with standard deviation of 2 minutes. A promotional poster for this bus system states that ``95\% of the time bus route \#25 lasts from \answer{50-2*\sqrt{20}} to \answer{50+2*\sqrt{20}} minutes.'' What numbers would you fill in the blanks with?
\end{problem}



\begin{problem}
 Let $X \sim Poi(9)$. Give a lower bound for $P( |X -\mu| \leq 5)$ using Chebyshev's Inequality.
\begin{hint}
\item $\sigma_X=\answer{3}$.
\item This is not a hint but an observation. If we used the PDF of X, we would get $\Pr\approx 0.9373$. Chebyshev's inequality is general and hence accounts for the possibility that the random variable at hand acts wildly. So, such an big gap with the outcome of the application of Chebyshev and actual answer is expected for nice random variables such as Poisson. 
\end{hint}
\begin{prompt}
$$\Pr( |X -\answer{9}| \leq 5)=\Pr(\answer{4} \leq X \leq \answer{14})=\Pr\left(\left|\frac{X-\mu}{\sigma}\right| \leq \answer{4/3}\right)\geq \answer{0.64}$$
\end{prompt}
\end{problem}


\begin{problem}
 Let $X \sim \mathcal N (100,15) $. Give a lower bound for $P( |X -\mu| \leq 20)$ using Chebyshev's Inequality.
\begin{hint}
This is not a hint but an observation. If we used the table for standard normal distribution, we would obtain, we would get $\Pr\approx 0.8176$. Again Normal distribution is really nice and hence, the Chebyshev bound is not really strict.
\end{hint}
\begin{prompt}
$$\Pr( |X -\answer{100}| \leq 20)=\Pr(\answer{80} \leq X \leq \answer{120})=\Pr\left(\left|\frac{X-\mu}{\sigma}\right| \leq \answer{20/15}\right)\geq \answer{0.4375}$$
\end{prompt}
\end{problem}

\begin{problem}
Let $X$ be a random variable with a finite expected value (mean) $\mu=E[X]$. Then {\bf Markov's inequality} states that, for any $t>0$, 
$$\Pr(X\geq t)\leq \frac{E(X)}{t}.$$ Suppose $X$ is the exponential random variable with parameter $\lambda=1$. Calculate the Markov bound for $\Pr(X\geq 3)$. 
\begin{hint}
If X is an exponential random variable with parameter $\lambda>0$, then $\mu_X=\sigma_X=\frac{1}{\lambda}$.
\end{hint}
\begin{prompt}%err
$$\Pr(X\geq 3)\leq \answer{1/3}.$$

\end{prompt}
\end{problem}


\begin{problem}
Suppose $X$ is the exponential random variable with parameter $\lambda=1$. Calculate Chebyshev bound for $\Pr(X\geq 3)$. 
\begin{hint}
\item If X is an exponential random variable with parameter $\lambda>0$, then $\mu_X=\sigma_X=\frac{1}{\lambda}$.
\item Original form of Chebyshev's bound is $\Pr(|(X-\mu)/\sigma|>k)\leq 1/k^2$. You might want to use this. 
\item when compared the previous problem, Chebyshev's bound is better. The reason is that Markov's inequality is a very primitive one. It would be fun to discuss why but we do not have time for this. 
\end{hint}
\begin{prompt}%err
$$\Pr(X\geq 3)\leq \answer{1/4}.$$

\end{prompt}
\end{problem}
\end{document}
