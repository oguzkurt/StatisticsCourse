\documentclass{ximera}
%% You can put user macros here
%% However, you cannot make new environments

\usepackage[letterpaper, total={6in, 8in}]{geometry}

\usepackage{tikz}
\usepackage{tkz-euclide}
\usetkzobj{all}

\tikzstyle geometryDiagrams=[ultra thick,color=blue!50!black]

%\usepackage{enumerate}
\usepackage{euler}

 
\usetikzlibrary{shapes,snakes}
\tikzset{
  dot hidden/.style={},
  line hidden/.style={},
  dot colour/.style={dot hidden/.append style={color=#1}},
  dot colour/.default=black,
  line colour/.style={line hidden/.append style={color=#1}},
  line colour/.default=black
}

%\begin{sagesilent}
def mydata(k):
    import numpy as np
    # Random test data
    np.random.seed(123)
    all_data = [np.random.normal(0, std, 100) for std in range(1, k)]
    return all_data

#boxplot
def seabox(my_data,mystr):
    import seaborn as sns
    sns.set_style("whitegrid")
    #tips = sns.load_dataset("tips")
    ax = sns.boxplot(x=my_data)
    ax.set(xlim=(0, None))    
    ax.get_figure().savefig(mystr,bbox_inches='tight')

#histogram
def myhist(k,b,m,M):
    from sage.plot.histogram import Histogram
    #k=data size
    # b= number of bins
    # random datA range
    a=histogram([randint(m,M) for _ in range(k)], bins=b)
    return a

#barchart & table
'''data1=[
['Zoo','giraffes', 'orangutans', 'monkeys'], ['SF Zoo', 20, 14, 23],['LA Zoo',12, 18, 29]
]'''
def mytb(data1,table_name,bar_name):
    import plotly.plotly as py
    import plotly.graph_objs as go
    from plotly.tools import FigureFactory as FF 
    from IPython.display import Image

    table1=FF.create_table(data1)
    py.iplot(table1,filename=table_name[:-4])
    py.image.save_as(table1, filename=table_name)
    Image(table_name)
    width=len(data1[0])
    height=len(data1)
    show(width)
    trace=[]
    d0=data1[0][1:width]
    show(d0)
    for i in range(height-1):
        show(i)
        d1=data1[i+1][1:width]
        show(d1)
        trace.append(
            go.Bar(
            x=d0,
            y=d1,
            name=data1[i+1][0]
                   )
                     )
    data = trace
    layout = go.Layout(
        barmode='group'
    )

    fig = go.Figure(data=data, layout=layout)
    py.iplot(fig, filename=bar_name[:-4])
    py.image.save_as(fig, filename=bar_name)
    Image(bar_name)
#piechart :
def mypie(labels,values,pie_name):
    import plotly.plotly as py
    import plotly.graph_objs as go
    from IPython.display import Image
    trace=go.Pie(labels=labels,values=values)
    py.iplot([trace], filename=data_name[:-4])
    py.image.save_as([trace],filename=pie_name)
    Image(pie_name)
    
\end{sagesilent}



\author{Oguz Kurt}

\title{Exam 2 problems}
\begin{document}
%feel free to change the title page
%

\begin{abstract}
\empty
\end{abstract}
\maketitle

\begin{problem}
The first problem is covered directly in HW.

\end{problem}

\begin{problem}

\begin{enumerate}
    \item A grasshopper hops each time with a probability of $0.99$. What is the probability that it will fail to hop at most 3 times in 100 tries?  
    
\begin{prompt}
$$X\sim Binomial(10,0.99)\Longrightarrow\Pr(X\leq 3)=\answer[tolerance=0.000000000000000000000000000000000000000000000000000000000000000000000000000000000000000000000000000000000000000000000000000000000000000000000000000000000000000000000000000000000000000000000000000001]{\sage{binomial(100,0)*(0.99)^0*(0.01)^100+binomial(100,1)*(0.99)^1*(0.01)^99+binomial(100,2)*(0.99)^2*(0.01)^98+binomial(100,3)*(0.99)^3*(0.01)^97}}$$
\end{prompt}
\item We know that a honey bee dies after stinging somebody. Suppose probability that a honeybee will sting somebody when felt in danger is $0.1$. What is the probability that it will die in {\bf at most} the 3rd encounter with someone posing danger? 

\begin{prompt}
$$Y\sim Geometric(0.1)\Longrightarrow\Pr(Y\leq 3)=\answer[tolerance=0.000000000000000000000000000000000000000000000000000000000000000000000000000000000000000000000000000000000000000000000000000000000000000000000000000000000000000000000000000000000000000000000000000001]{0.1+0.1*0.9+0.1*0.9^2}$$
\end{prompt}
\item A barrel has 5 poisonous snakes and 45 non-poisonous snakes. You pick 6 snakes randomly. What is the probability that you will pick at least one poisonous snake? Is it less than 1/2?

\begin{prompt}
$$Z\sim HyperGeo(50,5,6)\Longrightarrow\Pr(Z\geq 1)=1-\Pr(Z=0)=\answer[tolerance=0.000000000000000000000000000000000000000000000000000000000000000000000000000000000000000000000000000000000000000000000000000000000000000000000000000000000000000000000000000000000000000000000000000001]{\sage{1-(binomial(45,6))/(binomial(50,6))}}$$
\end{prompt}
\end{enumerate}
\end{problem}


%\newpage

\begin{problem}
{\bf (10 points each)} \\
\large Please, do NOT simplify your answers!
\begin{enumerate}
    \item The number of defective engines produced in a {\bf day} by a car manufacturer is a Poisson random variable with parameter 5. What is the probability that the factory will produce at most 3 defective engines in a {\bf 5-day-week}?

\begin{prompt}    
    $X\sim Poi(5)$ on an interval size of $1$ day. We use a $5-day$ interval. So, we need to use $Y\sim Poi(\answer{25})$ as our random variable and solve
    $$\Pr(Y\leq 3)=\answer[tolerance=0.000000000000000000000000000000000000000000000000000000000000000000000000000000000000000000000000001]{\sage{exp(-25)*(1+25+25^2/2+25^3/6)}}$$
\end{prompt}

    \item The number of non-engine defective car parts produced in a {\bf 5-day-week} by the same manufacturer is a Poisson random variable with parameter 15. What is the probability that the company will produce at least 3 cars with non-engine defects in a single day?

\begin{prompt}    
    $Z\sim Poi(15)$ on an interval size of $5$ days. We use a 
    $1-day$ interval. So, we need to use $W\sim Poi(\answer{3})$ as our random variable and solve
    $$\Pr(W\leq 3)=1-\Pr(W=2)-\Pr(W=1)-\Pr(W=2)=\answer[tolerance=0.000000000000000000000000000000000000000000000000000000000000000000000000000000000000000000000000001]{\sage{1-exp(-3)*(1+3+3^2/2)}}$$
\end{prompt}    

    \item {\bf (Bonus)} What is the probability that the company will produce exactly 10 defective (engine or non-engine) cars in a {\bf 5-day-week}?

\begin{prompt}    
We count the total number of defects in a 5-day interval. So, we need to use a random variable $D=Y+Z$ where $Y\sim Poi(25)$ and $Z\sim Poi(15)$ on the same 5-day interval size. Hence, $D\sim Poi(\answer{25+15})$. The answer then is
    $$\Pr(Z=10)=\answer[tolerance=0.000000000000000000000000000000000000000000000000000000000000000000000000000000000000000000000000001]{\sage{exp(-40)*40^10/factorial(10)}}$$
\end{prompt}    


\end{enumerate}
\end{problem}



\begin{problem} { \bf (10 points each)} 
Let X be a random variable counting the number of dots on a fair die after a single roll and Y be the total number of dots after 3 independent rolls of the same die.  Calculate the following:
\\
{\bf Hint:} You might want to first calculate the mean and variance of $X$ in the extra space below. Then use the fact that $Y=X_1+X_2+X_3$ where $X_i$ is the number of dots in the $i^{\text{th}}$ roll of the die.


\begin{explanation}
\begin{itemize}
    \item $Y\sim X_1+X_2+X_3$
    \item We need to calculate the mean and variance of $X$ first.
 
 $E[X]=\answer{21/6}$ and $Var(X)=\answer{((1-3.5)^2+(2-3.5)^2+(3-3.5)^2+(4-3.5)^2+(5-3.5)^2+(6-3.5)^2)/6}$
    \item We know $E[A+B]=E[A]+E[B]$ for all random variables.
    \item We know $Cov(A,B)=0$ whenever A and B are independent.
    \item We know $Cov(A,A)=Var(A)$ for any random variable.
    \item We know
$$Var[X_1+X_2+X_3]=Var[X_1]+Var[X_2]+Var[X_3]+2Cov(X_1,X_2)+2Cov(X_1,X_3)+2Cov(X_2,X_3)   $$
\end{itemize}
\end{explanation} 

\begin{enumerate}
    \item $\mathbb E[Y]=\answer{3*21/6}$
    
    \item $\mathbb Var[Y]=\answer{3*(1-3.5)^2+(2-3.5)^2+(3-3.5)^2+(4-3.5)^2+(5-3.5)^2+(6-3.5)^2)/6}$


    \item {\bf (Bonus)} Is $\mathbb Var[Y]=\mathbb Var[3X]$? Why?
    \begin{multipleChoice}
        \choice{True}
        \choice[correct]{False}
    \end{multipleChoice}
    because $Var(Y)=\answer{3}\cdot Var(X)$ but $Var(3X)=\answer{9}\cdot Var(X)$.
    \vspace{2cm}
\end{enumerate}
\end{problem}
\end{document}
