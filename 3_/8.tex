\documentclass{ximera}
%% You can put user macros here
%% However, you cannot make new environments

\usepackage[letterpaper, total={6in, 8in}]{geometry}

\usepackage{tikz}
\usepackage{tkz-euclide}
\usetkzobj{all}

\tikzstyle geometryDiagrams=[ultra thick,color=blue!50!black]

%\usepackage{enumerate}
\usepackage{euler}

 
\usetikzlibrary{shapes,snakes}
\tikzset{
  dot hidden/.style={},
  line hidden/.style={},
  dot colour/.style={dot hidden/.append style={color=#1}},
  dot colour/.default=black,
  line colour/.style={line hidden/.append style={color=#1}},
  line colour/.default=black
}

%\begin{sagesilent}
def mydata(k):
    import numpy as np
    # Random test data
    np.random.seed(123)
    all_data = [np.random.normal(0, std, 100) for std in range(1, k)]
    return all_data

#boxplot
def seabox(my_data,mystr):
    mydata.sort()
    import seaborn as sns
    sns.set_style("whitegrid")
    #tips = sns.load_dataset("tips")
    ax = sns.boxplot(x=my_data)
    ax.set(xlim=(0, len(mydata)+10))    
#    ax.get_figure().savefig(mystr,bbox_inches='tight')
    
    
def seahist(data_list,bin,mystr):
    import matplotlib.pyplot as plt    
    import seaborn as sns
    ax=sns.distplot(data_list, color='red',bins=bin, kde=False)
    ax.set(xlim=(0, None))    
    ax.get_figure().savefig(mystr,bbox_inches='tight')
    plt.savefig(mystr)
    plt.clf()
    
    #histogram
def myhist(k,b,m,M):
    from sage.plot.histogram import Histogram
    #k=data size
    # b= number of bins
    # random datA range
    a=histogram([randint(m,M) for _ in range(k)], bins=b)
    return a

#barchart & table
'''data1=[
['Zoo','giraffes', 'orangutans', 'monkeys'], ['SF Zoo', 20, 14, 23],['LA Zoo',12, 18, 29]
]'''
def mytb(data1,table_name,bar_name):
    import plotly.plotly as py
    import plotly.graph_objs as go
    from plotly.tools import FigureFactory as FF 
    from IPython.display import Image

    table1=FF.create_table(data1)
    py.iplot(table1,filename=table_name[:-4])
    py.image.save_as(table1, filename=table_name)
    Image(table_name)
    width=len(data1[0])
    height=len(data1)
    show(width)
    trace=[]
    d0=data1[0][1:width]
    show(d0)
    for i in range(height-1):
        show(i)
        d1=data1[i+1][1:width]
        show(d1)
        trace.append(
            go.Bar(
            x=d0,
            y=d1,
            name=data1[i+1][0]
                   )
                     )
    data = trace
    layout = go.Layout(
        barmode='group'
    )

    fig = go.Figure(data=data, layout=layout)
    py.iplot(fig, filename=bar_name[:-4])
    py.image.save_as(fig, filename=bar_name)
    Image(bar_name)
#piechart :
def mypie(labels,values,pie_name):
    import plotly.plotly as py
    import plotly.graph_objs as go
    from IPython.display import Image
    trace=go.Pie(labels=labels,values=values)
    py.iplot([trace], filename=data_name[:-4])
    py.image.save_as([trace],filename=pie_name)
    Image(pie_name)
    
\end{sagesilent}



\title{Sampling Proportion/Mean Distributions - HW8}
\author{Oguz Kurt}

\begin{document}
\begin{abstract}
\empty
\end{abstract}
\maketitle
\begin{sagesilent}
def c(n,k):
    if 0<=k and k <=n:
        return binomial(n,k)
    else:
        return "error"

def p(n,k):
    if 0<=k and k <=n:
        return binomial(n,k)*factorial(k)
    else:
        return "error"
def m(n,mylist):
    l=len(mylist)
    return factorial(n)/(prod([factorial(mylist[i]) for i in range(l)]))
\end{sagesilent}

%\geogebra{sSsy52bG}


\section*{Syntax}
Here is a reminder of the way to input some numbers. Please, use the parsing power of your browser instead of using a calculator.

\hspace{0.5cm}


\begin{tabular}{c|c}
Operation & Syntax  
\\
\hline
$a+b$ & {\color{red} \verb!a+b!} 
\\
\hline
$a- b$ & {\color{red} \verb!a-b!} 
\\
\hline
$a\cdot b$ & {\color{red} \verb!a*b! } 
\\
\hline
$a/b=\frac{a}{b}$ & {\color{red} \verb!a/b!}
\\
\hline
$a^b$ & {\color{red} \verb|a**b| or \verb|a^b|} 
\\
\hline
$\sqrt{a}$ & {\color{red} \verb|sqrt(a)| or \verb|a**(1/2)|} 
\\
\hline
$a!$ & {\color{red} \verb|a!|} 
\\
\hline
$\binom{a}{b}$ & {\color{red} \verb|a!/(b!*(a-b)!)|}
\\
\hline
$P_k^n$ & {\color{red} \verb|a!/(a-b)!|}
\\
\hline
$e^x$ & {\color{red} \verb|exp(x)|}
\\
\end{tabular}

\section*{Sampling Proportion/Mean Distributions}
\subsection*{Sampling Proportion Distributions}
Suppose we know, for a given experiment, the proportion of success in a population is $p_0$. If we pick a sample of size n from this population, then the proportion of success $\hat p$ in a given sample of size of  n is a random variable. To understand this, we can treat $Bernoulli(p_0)$-random variable $X_i$ such that $\Pr(X_i=1)=p_0$ and $\Pr(X_i=0)=1-p_0$, the proportion $\hat p$ can be seen as $\hat p=\frac{X_1+X_2+\ldots+X_n}{n}$ of where $Y=X_1+X_2+\ldots+X_n\sim Binomial(n,p_0)$.  Since we know $\mu_Y=np_0$ and $Var(Y)=np_0(1-p_0)$ for binomial random variables, we get 
$$\mu_{\hat p}=\frac{\mu_Y}{n}=p_0 \mbox{ and } \sigma_{\hat p}=\sigma_{Y/n}=\sqrt{\frac{Var(Y)}{n^2}}=\sqrt{\frac{p_0(1-p_0)}{n}}$$

We further know that $\hat p\approx \mathcal N(p_0,\sqrt{\frac{p_0(1-p_0)}{n}})$ gives a nice approximation if $np_0\geq 5$ and $n(1-p_0)\geq 5$ are both satisfied. 

\begin{problem}
The company JCrew advertises that 95\% of its online orders ship within two working
days. You select a random sample of 200 of the 10,000 orders received over the past month to
audit. The audit reveals that 180 of these orders shipped on time. This is a categorical problem
(sample proportion).
\begin{enumerate}
    \item Sample proportion of orders shipped on time is $\hat p=\answer{180/200}$
 
    \item Does the sample data satisfy conditions necessary for the sample proportion to follow an
approximately normal distribution?
\begin{multipleChoice}
    \choice{No}
    \choice[correct]{Yes}
\end{multipleChoice}
{\bf Reason:} $np_0=\answer{\sage{200*0.95}}$
\wordChoice{ \choice{$<5$} \choice[correct]{$\geq 5$}}
and $n(1-p_0)=\answer{\sage{200*0.05}}$
\wordChoice{ \choice{$<5$} \choice[correct]{$\geq 5$}}
    
    \item What is the mean and standard deviation of the sample distribution assuming it's normal?
    $\mu_{\hat p}=\answer{0.95}$ and $\sigma_{\hat p}=\answer[tolerance=0.0001]{\sage{sqrt(.95*.05/200)}}$

    \item . If JCrew really ships 95\% of its orders on time, what is probability that the proportion in a
random sample of 200 orders is as small or smaller as the proportion in the audit?
$$\Pr(\hat p \leq 0.9)=\Pr(Z\leq \answer[tolerance=0.005]{-3.33})=\answer[tolerance=0.00001]{0.0006}$$
\item If we treated the problem as a binomial, how would the problem be set up? That is, what
would we want to find the probability of?
\begin{explanation}
$Y=n\hat p \approx Binomial(\answer{200},\answer{.95})$.
Hence, we need to calculate $\Pr(Y\leq \answer{180})$ in the place of part (d). 
\end{explanation}
\end{enumerate}
\end{problem}

\begin{problem}
A Gallup poll found that for those Americans who have lost weight, 31\% believed the most
effective strategy involved exercise. What is the probability that from random sample of 300
Americans the sample proportion falls between 29\% and 33\%? Categorical (sample proportion).  

\begin{explanation}
\begin{itemize} 
    \item Sample data \wordChoice{\choice[correct]{does}\choice{does not}} satisfy conditions necessary for the sample proportion to follow an
approximately normal distribution because $np_0=\answer{\sage{300*.31}}$ is 
\wordChoice{ \choice{less than} \choice[correct]{more than}} 5.
and $n(1-p_0)=\answer{\sage{300*.69}}$
\wordChoice{ \choice{$<5$} \choice[correct]{$\geq 5$}}
    
    \item $\mu_{\hat p}=\answer{0.31}$ and $\sigma_{\hat p}=\answer[tolerance=0.0001]{\sage{sqrt(.31*.69/300)}}$

    \item We need to calculate:    
\begin{eqnarray*}
\Pr(0.29\leq \hat p \leq 0.33)
&=&
\Pr(\hat p \leq 0.33) -\Pr(\hat p \leq 0.29)
\\
&=&
\Pr(Z \leq \answer[tolerance=0.001]{\sage{(0.33 - 0.31)/sqrt(.31*.69/300)} }) -\Pr(Z \leq \answer[tolerance=0.001]{(0.29 - 0.31)/0.0267 })
\\
&=& 
\phi(\answer[tolerance=0.001]{\sage{(0.33 - 0.31)/sqrt(.31*.69/300)} })-\phi(\answer[tolerance=0.001]{\sage{(0.29 - 0.31)/sqrt(.31*.69/300)} })
\\
&=&
\answer[tolerance=0.00005]{0.5468}
\end{eqnarray*}
\end{itemize}
\end{explanation}
\end{problem}

\subsection*{Sampling Mean Distributions}
Suppose we have a data over a population with mean $\mu$ and standard deviation $\sigma$ over a population. If we pick a sample of size n from this population, the mean $\bar x$ of such a sample is a random variable 
$$\bar{x}=\frac{X_1+X_2+\ldots+X_n}{n}$$
where $X_i$ is the $i^{th}$ random pick from this population. We then know
$$\mu_{\bar x}=\mu \mbox{ and } \sigma_{\bar x}=\frac{\sigma}{\sqrt{n}}.$$

We further know that $\bar x\approx \mathcal N(\mu,\frac{\sigma}{\sqrt{n}})$ gives a nice approximation if the population data is distributed normally or $n\geq 30$.

\begin{remark}
Please, note that we tried to represent sample proportion distributions and sample mean distributions using the same notation intentionally. Our reasoning is quite simple. Sample proportion distributions is a special case of sample mean distributions. We usually use it if our original data is categorical in nature (success/failure) while sample mean distributions are used on numerical data. 
\end{remark}

\begin{problem}
A bottling company uses a machine to fill the bottles with olive oil. The bottles are designed
to contain 475 milliliters (ml). In fact, the contents vary according to a normal distribution with
a mean of 473 ml and standard deviation of 3 ml. Quantitative (sample mean). 

    \begin{enumerate}
        \item What is the distribution, mean, and distribution of the sample mean of six randomly
selected bottles?

        \begin{itemize}
        \item Distribution is approximately \wordChoice{\choice{exponential} \choice{uniform} \choice[correct]{normal}}.
%        \item Sample mean should be normal with a mean$\mu_{\bar{x}} = \answer{473}$.
        \item Standard deviation is $\sigma_{\bar x}=\answer[tolerance=0.0001]{\sage{3/sqrt(6)}}$
        \end{itemize}        
        
        \item  What is the probability that the mean of six bottles is less than 470 ml?
        $\Pr(\bar X<470)=\phi(\answer[tolerance=0.0001]{\sage{(470-473)/(3/sqrt(6))}})=\answer[tolerance=0.0001]{0.0069}$
        \item  What is probability that the mean of six bottles is more than 475 ml?
        $\Pr(\bar X>475)=1-\phi(\answer[tolerance=0.0001]{\sage{(475-473)/(3/sqrt(6))}})=\answer[tolerance=0.0001]{0.0505}$
    \end{enumerate}
\end{problem}
\begin{problem}
PSF which operates and manages car rentals for PS employees found
that the tire lifetime for their vehicles has a mean of 50,000 miles and standard deviation of 3500 miles. Quantitative (sample mean).
\begin{enumerate}
    \item What would be the distribution, mean and standard deviation for mean lifetime of a random sample of 50 vehicles?
    $$\bar x \approx \mathcal N(\mu=\answer{50000},         \sigma=\answer[tolerance=0.0001]{\sage{3500/sqrt(50)}})$$
    \item  What is the probability that the sample mean lifetime for these 50 vehicles exceeds 52,000?
    $$\Pr(\bar x\geq 52,000 )= \Pr(Z\geq \answer[tolerance=0.0001]{\sage{(52000-50000)/(3500/sqrt(50))}}=1-\phi(\answer[tolerance=0.0001]{\sage{(52000-50000)/(3500/sqrt(50))}})=\answer[tolerance=0.00007]{0.00003}$$

\end{enumerate}
\end{problem}
\end{document}
 