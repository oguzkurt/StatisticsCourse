\documentclass{ximera}
%% You can put user macros here
%% However, you cannot make new environments

\usepackage[letterpaper, total={6in, 8in}]{geometry}

\usepackage{tikz}
\usepackage{tkz-euclide}
\usetkzobj{all}

\tikzstyle geometryDiagrams=[ultra thick,color=blue!50!black]

%\usepackage{enumerate}
\usepackage{euler}

 
\usetikzlibrary{shapes,snakes}
\tikzset{
  dot hidden/.style={},
  line hidden/.style={},
  dot colour/.style={dot hidden/.append style={color=#1}},
  dot colour/.default=black,
  line colour/.style={line hidden/.append style={color=#1}},
  line colour/.default=black
}

%\begin{sagesilent}
def mydata(k):
    import numpy as np
    # Random test data
    np.random.seed(123)
    all_data = [np.random.normal(0, std, 100) for std in range(1, k)]
    return all_data

#boxplot
def seabox(my_data,mystr):
    import seaborn as sns
    sns.set_style("whitegrid")
    #tips = sns.load_dataset("tips")
    ax = sns.boxplot(x=my_data)
    ax.set(xlim=(0, None))    
    ax.get_figure().savefig(mystr,bbox_inches='tight')

#histogram
def myhist(k,b,m,M):
    from sage.plot.histogram import Histogram
    #k=data size
    # b= number of bins
    # random datA range
    a=histogram([randint(m,M) for _ in range(k)], bins=b)
    return a

#barchart & table
'''data1=[
['Zoo','giraffes', 'orangutans', 'monkeys'], ['SF Zoo', 20, 14, 23],['LA Zoo',12, 18, 29]
]'''
def mytb(data1,table_name,bar_name):
    import plotly.plotly as py
    import plotly.graph_objs as go
    from plotly.tools import FigureFactory as FF 
    from IPython.display import Image

    table1=FF.create_table(data1)
    py.iplot(table1,filename=table_name[:-4])
    py.image.save_as(table1, filename=table_name)
    Image(table_name)
    width=len(data1[0])
    height=len(data1)
    show(width)
    trace=[]
    d0=data1[0][1:width]
    show(d0)
    for i in range(height-1):
        show(i)
        d1=data1[i+1][1:width]
        show(d1)
        trace.append(
            go.Bar(
            x=d0,
            y=d1,
            name=data1[i+1][0]
                   )
                     )
    data = trace
    layout = go.Layout(
        barmode='group'
    )

    fig = go.Figure(data=data, layout=layout)
    py.iplot(fig, filename=bar_name[:-4])
    py.image.save_as(fig, filename=bar_name)
    Image(bar_name)
#piechart :
def mypie(labels,values,pie_name):
    import plotly.plotly as py
    import plotly.graph_objs as go
    from IPython.display import Image
    trace=go.Pie(labels=labels,values=values)
    py.iplot([trace], filename=data_name[:-4])
    py.image.save_as([trace],filename=pie_name)
    Image(pie_name)
    
\end{sagesilent}



\title{Basic Definitions}
\author{Oguz Kurt}
\begin{abstract}
   Basic definitions needed to start statistics. Copied from OpenStax.org to help with the first HW.
\end{abstract}

%\outcome{Identify, distinguish natural, whole numbers and integers.}
%\outcome{Use natural, whole numbers and integers in operations.}
%\outcome{Learn the impact of different operations on natural, whole numbers and integers.}

\begin{document}
\maketitle

\begin{itemize}
\item \textbf{Variable:}
a characteristic of interest for each person or object in a population
\item \textbf{Categorical Variable:}
variables that take on values that are names or labels. Nominal is non-number and ordinal is a number variable that is categorical.
\item \textbf{Numerical Variable:}
variables that take on values that are indicated by numbers. If the order of the input matters but not the proportion, it is an interval type variable. Otherwise, it is a ratio type variable. 
\item \textbf{Data:}
a set of observations (a set of possible outcomes); most data can be put into two groups: qualitative (an attribute whose value is indicated by a label) or quantitative (an attribute whose value is indicated by a number). Quantitative data can be separated into two subgroups: discrete and continuous. Data is discrete if it is the result of counting (such as the number of students of a given ethnic group in a class or the number of books on a shelf). Data is continuous if it is the result of measuring (such as distance traveled or weight of luggage)
\item \textbf{Population:}
all individuals, objects, or measurements whose properties are being studied
\item \textbf{Sample:}
a subset of the population studied
\item \textbf{Representative Sample:}
a subset of the population that has the same characteristics as the population
\item \textbf{Proportion:}
the number of successes divided by the total number in the sample
\item \textbf{Parameter:}
a number that is used to represent a population characteristic and that generally cannot be determined easily
\item \textbf{Average:}
also called \textbf{mean}; a number that describes the central tendency of the data
\item \textbf{Probability:}
a number between zero and one, inclusive, that gives the likelihood that a specific event will occur
\item \textbf{Statistic:}
a numerical characteristic of the sample; a statistic estimates the corresponding population parameter.
\end{itemize}
\end{document}
