\documentclass{ximera}
%% You can put user macros here
%% However, you cannot make new environments

\usepackage[letterpaper, total={6in, 8in}]{geometry}

\usepackage{tikz}
\usepackage{tkz-euclide}
\usetkzobj{all}

\tikzstyle geometryDiagrams=[ultra thick,color=blue!50!black]

%\usepackage{enumerate}
\usepackage{euler}

 
\usetikzlibrary{shapes,snakes}
\tikzset{
  dot hidden/.style={},
  line hidden/.style={},
  dot colour/.style={dot hidden/.append style={color=#1}},
  dot colour/.default=black,
  line colour/.style={line hidden/.append style={color=#1}},
  line colour/.default=black
}

%\begin{sagesilent}
def mydata(k):
    import numpy as np
    # Random test data
    np.random.seed(123)
    all_data = [np.random.normal(0, std, 100) for std in range(1, k)]
    return all_data

#boxplot
def seabox(my_data,mystr):
    import seaborn as sns
    sns.set_style("whitegrid")
    #tips = sns.load_dataset("tips")
    ax = sns.boxplot(x=my_data)
    ax.set(xlim=(0, None))    
    ax.get_figure().savefig(mystr,bbox_inches='tight')

#histogram
def myhist(k,b,m,M):
    from sage.plot.histogram import Histogram
    #k=data size
    # b= number of bins
    # random datA range
    a=histogram([randint(m,M) for _ in range(k)], bins=b)
    return a

#barchart & table
'''data1=[
['Zoo','giraffes', 'orangutans', 'monkeys'], ['SF Zoo', 20, 14, 23],['LA Zoo',12, 18, 29]
]'''
def mytb(data1,table_name,bar_name):
    import plotly.plotly as py
    import plotly.graph_objs as go
    from plotly.tools import FigureFactory as FF 
    from IPython.display import Image

    table1=FF.create_table(data1)
    py.iplot(table1,filename=table_name[:-4])
    py.image.save_as(table1, filename=table_name)
    Image(table_name)
    width=len(data1[0])
    height=len(data1)
    show(width)
    trace=[]
    d0=data1[0][1:width]
    show(d0)
    for i in range(height-1):
        show(i)
        d1=data1[i+1][1:width]
        show(d1)
        trace.append(
            go.Bar(
            x=d0,
            y=d1,
            name=data1[i+1][0]
                   )
                     )
    data = trace
    layout = go.Layout(
        barmode='group'
    )

    fig = go.Figure(data=data, layout=layout)
    py.iplot(fig, filename=bar_name[:-4])
    py.image.save_as(fig, filename=bar_name)
    Image(bar_name)
#piechart :
def mypie(labels,values,pie_name):
    import plotly.plotly as py
    import plotly.graph_objs as go
    from IPython.display import Image
    trace=go.Pie(labels=labels,values=values)
    py.iplot([trace], filename=data_name[:-4])
    py.image.save_as([trace],filename=pie_name)
    Image(pie_name)
    
\end{sagesilent}



\title{Counting and Probability 1}
\author{Oguz Kurt}

\begin{document}
\begin{abstract}
We introduce the syntax and some key theorems we discussed in class. No examples or proofs will be presented here since we have done those in class.
\end{abstract}
\maketitle


\begin{sagesilent}
def c(n,k):
    if 0<=k and k <=n:
        return binomial(n,k)
    else:
        return "error"

def p(n,k):
    if 0<=k and k <=n:
        return binomial(n,k)*factorial(k)
    else:
        return "error"

\end{sagesilent}

\section*{Syntax}
In this HW, you are expected to solve problems of counting. Before we start, here is some help as to how to answer the problems. You can of course calculate the answer using a calculator and enter it to the corresponding answer box but there is an easier way for you. This table is to help you. Do not calculate the answer in your mind or using a calculator. Instead, use the syntax and let the answer box do it for you:

\hspace{0.5cm}

\begin{exercise}
\begin{tabular}{c|c|c|c}
Operation & Syntax & Example & Answer 
\\
\hline
$a+b$ & {\color{red} a+b} & $3+5$ & $\answer{\sage{3+5} }$
\\
\hline
$a- b$ & {\color{red} a-b} & $3-5$ & $\answer{\sage{3-5} }$
\\
\hline
$a\cdot b$ & {\color{red} a*b} & $3\times 5$ & $\answer{\sage{3*5} }$
\\
\hline
$a/b=\frac{a}{b}$ & {\color{red} a/b} & $\frac{3}{5}$ & $\answer{\sage{3/5} }$
\\
\hline
$a^b$ & {\color{red} a**b} & $3^5$ & $\answer{\sage{3**5} }$
\\
\hline
$\sqrt{a}$ & {\color{red} sqrt(a) or a**(1/2)} & $\sqrt{5}$ & $\answer{\sage{sqrt(5)} }$
\\
\hline
$a!$ & {\color{red} a!} & $50!$ & $\answer{\sage{factorial(50)} }$
\\
\hline
$\binom{a}{b}$ & {\color{red} a!/(b!*(a-b)!)} & $\binom{50}{45}$ & $\answer{\sage{factorial(50)/factorial(45)/factorial(5)}}=\answer{\sage{binomial(50,45)}}$ 
\\
\hline
$P_k^n$ & {\color{red} a!/(a-b)!} & $P_{30}^{100}$ & $\answer{\sage{binomial(100,30)*factorial(30)} }$ 
\\
\end{tabular}
\end{exercise}

\hspace{1cm}

For some of the problems, I do care about the number answer. Then you will have to first calculate the answer and then input the output into the answer box. The problem will specifically ask you to complete the "calculations".

\section*{Some Basic Theorems and Definitions for Counting}

We assume your knowledge of sets and hence skip them here.

\begin{theorem}[Addition Rule]
Suppose A and B are \textbf{mutually exclusive} or \textbf{mutually disjoint}  events. Then the total number of outcomes is $|A\cup B|=|A|+|B|$
\end{theorem}

\begin{example}
Suppose we roll a pair of dice. How many ways are there to get the total of 8 or 10?

\begin{explanation}
We define A as the event that the total is 8 and B as the event that the total is 10. Then 

$$A=\{(2,6),~(3,5),~(4,4),~(5,3),~(6,2)\}$$
$$B=\{(4,6),~(5,5),~(6,4)\}$$

It is clear that $A\cap B=\emptyset$ as no pair can have the total of 8 and 10 at the same time.

Our main event E is the set of all pairs of dice with a total of 8 or 10. We get $E=A\cup B$ and A and B are mutually exclusive or mutually disjoint. Hence, by addition rule, we get

$$|A\cup B|=|A|+|B|=\answer{5}+\answer{3}=\answer{8}$$


\end{explanation}
\end{example}





\begin{problem}
$\answer{\sage{factorial(500)/factorial(199)}}$
\end{problem}

\end{document}

