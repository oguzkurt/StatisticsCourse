\documentclass{ximera}
%% You can put user macros here
%% However, you cannot make new environments

\usepackage[letterpaper, total={6in, 8in}]{geometry}

\usepackage{tikz}
\usepackage{tkz-euclide}
\usetkzobj{all}

\tikzstyle geometryDiagrams=[ultra thick,color=blue!50!black]

%\usepackage{enumerate}
\usepackage{euler}

 
\usetikzlibrary{shapes,snakes}
\tikzset{
  dot hidden/.style={},
  line hidden/.style={},
  dot colour/.style={dot hidden/.append style={color=#1}},
  dot colour/.default=black,
  line colour/.style={line hidden/.append style={color=#1}},
  line colour/.default=black
}

%\begin{sagesilent}
def mydata(k):
    import numpy as np
    # Random test data
    np.random.seed(123)
    all_data = [np.random.normal(0, std, 100) for std in range(1, k)]
    return all_data

#boxplot
def seabox(my_data,mystr):
    import seaborn as sns
    sns.set_style("whitegrid")
    #tips = sns.load_dataset("tips")
    ax = sns.boxplot(x=my_data)
    ax.set(xlim=(0, None))    
    ax.get_figure().savefig(mystr,bbox_inches='tight')

#histogram
def myhist(k,b,m,M):
    from sage.plot.histogram import Histogram
    #k=data size
    # b= number of bins
    # random datA range
    a=histogram([randint(m,M) for _ in range(k)], bins=b)
    return a

#barchart & table
'''data1=[
['Zoo','giraffes', 'orangutans', 'monkeys'], ['SF Zoo', 20, 14, 23],['LA Zoo',12, 18, 29]
]'''
def mytb(data1,table_name,bar_name):
    import plotly.plotly as py
    import plotly.graph_objs as go
    from plotly.tools import FigureFactory as FF 
    from IPython.display import Image

    table1=FF.create_table(data1)
    py.iplot(table1,filename=table_name[:-4])
    py.image.save_as(table1, filename=table_name)
    Image(table_name)
    width=len(data1[0])
    height=len(data1)
    show(width)
    trace=[]
    d0=data1[0][1:width]
    show(d0)
    for i in range(height-1):
        show(i)
        d1=data1[i+1][1:width]
        show(d1)
        trace.append(
            go.Bar(
            x=d0,
            y=d1,
            name=data1[i+1][0]
                   )
                     )
    data = trace
    layout = go.Layout(
        barmode='group'
    )

    fig = go.Figure(data=data, layout=layout)
    py.iplot(fig, filename=bar_name[:-4])
    py.image.save_as(fig, filename=bar_name)
    Image(bar_name)
#piechart :
def mypie(labels,values,pie_name):
    import plotly.plotly as py
    import plotly.graph_objs as go
    from IPython.display import Image
    trace=go.Pie(labels=labels,values=values)
    py.iplot([trace], filename=data_name[:-4])
    py.image.save_as([trace],filename=pie_name)
    Image(pie_name)
    
\end{sagesilent}



\title{Counting and Probability II-- Homework 3}
\author{Oguz Kurt}

\begin{document}
\begin{abstract}
In this HW, you are expected to solve basic probability problems. To succeed, you are expected to know basics of counting, probability and, (relative) frequency. While we have not defined it, we will later use the knowledge of frequencies in learning probability distribution function (pdf) and cumulative distribution function (cdf).
\end{abstract}
\maketitle

\begin{sagesilent}
def c(n,k):
    if 0<=k and k <=n:
        return binomial(n,k)
    else:
        return "error"

def p(n,k):
    if 0<=k and k <=n:
        return binomial(n,k)*factorial(k)
    else:
        return "error"
def m(n,mylist):
    l=len(mylist)
    return factorial(n)/(prod([factorial(mylist[i]) for i in range(l)]))
\end{sagesilent}

\section*{Syntax}
In this HW, you are expected to solve problems of basic probability. Here is a reminder of the way to input some numbers. Please, use the parsing power of your browser instead of using a calculator.

\hspace{0.5cm}


\begin{tabular}{c|c}
Operation & Syntax  
\\
\hline
$a+b$ & {\color{red} \verb!a+b!} 
\\
\hline
$a- b$ & {\color{red} \verb!a-b!} 
\\
\hline
$a\cdot b$ & {\color{red} \verb!a*b! } 
\\
\hline
$a/b=\frac{a}{b}$ & {\color{red} \verb!a/b!}
\\
\hline
$a^b$ & {\color{red} \verb|a**b| or \verb|a^b|} 
\\
\hline
$\sqrt{a}$ & {\color{red} \verb|sqrt(a)| or \verb|a**(1/2)|} 
\\
\hline
$a!$ & {\color{red} \verb|a!|} 
\\
\hline
$\binom{a}{b}$ & {\color{red} \verb|a!/(b!*(a-b)!)|}
\\
\hline
$P_k^n$ & {\color{red} \verb|a!/(a-b)!|}
\\
\end{tabular}

\hspace{1cm}

\begin{problem}
The probability that a particular type of rifle will fire is $p$. We test the rifle $k$ times.
What is the probability it will fire in every one of $k$ independent trials? 

\begin{explanation}
\begin{sagesilent}
p=var('p')
k=var('k')
\end{sagesilent}
\begin{equation*} 
\Pr(k \text{ successes} ) = \answer[given]{\sage{p^k}}
\end{equation*}

\end{explanation}
\end{problem}

\begin{problem}
The probability that a particular type of rifle will fire is $p$. We test the rifle $k$ times.
What is the probability it will fire exactly $5$ times in $k$ independent trials? 

\begin{explanation}
\begin{sagesilent}
q=var('q')
l=var('l')
\end{sagesilent}
\begin{equation*} 
\Pr(5 \text{ successes in  } \ell \text{ trials } ) = \answer[given]{\sage{binomial(l,5)*q^5*(1-q)^(l-5)}} 
\end{equation*}
\end{explanation}
\end{problem}

\end{document}


\begin{problem}
How many ways are there to select 5 spades, 4 hearts, 3 diamonds, and 1 club to make a bridge player's hand?

\begin{explanation}
   Each suit has $\answer{13}$ cards. So, the answer is 
   $$\answer{\sage{c(13,5)*c(13,4)*c(13,3)*c(13,1)}}.$$
\end{explanation}
\end{problem}


\begin{problem}
Forty equally skilled teams play a tournament in which every team plays every other team exactly once, and there are not ties.

\begin{enumerate}
\item  How many different games were played?
$$
\answer{\sage{c(40,2)}}
$$
\item  How many different possible outcomes for these games are
there?
%Each of the 780 games has two possible outcomes, so the total number of different outcomes is
$$
\answer{2}^{\answer{780}}
$$
\item  How many different ways are there for each team to win a different number of games?

%If each team wins a different number of games, this corresponds to a unique ordering of the teams.There are 
$$
\answer{\sage{factorial(40)}}
$$ 
%such orderings.
\end{enumerate}
\end{problem}

\begin{problem} 
A football team ends the regular season with an 11-5 record. How many different sequences of wins and losses can lead to this outcome?

\begin{explanation}
   We are writing a 16-letter word using the alphabet $\{w,\ell\}$ by using $w$ 11 times and $\ell$ 5 times. So, the answer is 
   $$\answer{\sage{c(16,11)}}.$$
\end{explanation}
\end{problem}

\begin{problem} 
How many different terms of the form $x^ix^jx^k$ are equal to $x^{10}?$

\begin{explanation}
We count the number of non-negative integer triples $(i,j,k)$ such that $i+j+k=10$. So, the answer is 
   $$\answer{\sage{c(10+2,2)}}.$$
\end{explanation}
\end{problem}

\begin{problem}
How many permutations are there of the letters in the word
  \verb!BOOKKEEPER! in which the two \verb!O!s and the two
\verb!K!s are adjacent (but not necessarily next to each other).
\begin{hint}
Treat \verb!OO! and \verb!KK! as single letters.
\end{hint}
$$
\answer{
\sage{m(8,[1,1,1,3,1,1])
}
}
$$
\end{problem}

\begin{problem} 
At a small zoo, you are in charge of a lion, a tiger, a cheetah, and panther, and a jaguar, all in separate cages. Overnight the cages are opened by vandals, and the cats all escape. The next morning you put food in the empty cages and sure enough, the cats settle down, one per cage, for breakfast. All permutations are equally likely. What is the number of ways that one cat has returned to its original cage while two pairs have swapped cages? 

\begin{explanation}
   $$\answer{\sage{c(5,1)*c(4,2)/2}}.$$
\end{explanation}
\end{problem}

\begin{problem} 
 Star Pizza offers three choices for the diameter,
two choices for crust thickness, and 12 choices of topping.   Suppose
that you limit yourself to 5 or fewer toppings.   How many different
pizzas can you construct?
\begin{hint}
    \item[$(1)$] You first need to calculate the number of ways to select 5 or fewer toppings!
        \begin{hint}
            \item[$(1a)$] You need to use your knowledge of combinations and Addition Rule for this!
            \item[$(1b)$] You need to find the number of ways to select k=0,1,2,3,4,5 toppings out of 12 toppings first. 
        \end{hint}
    \item[$(2)$] Next, you need to use Multiplication Rule to get all possible kinds of pizzas!
        \begin{hint} $(2a)$
            How many different types of diameters, crust types and choices of at most 5 toppings do you have? You need to answer this question for each group!
        \end{hint}
\end{hint}

\begin{explanation}
    Number of all kinds of pizzas with the aforementioned possible choices is $$\answer{\sage{3*2*(binomial(12,0)+binomial(12,1)+binomial(12,2)+binomial(12,3)+binomial(12,4)+binomial(12,5))} }$$
\end{explanation}
\end{problem}



\begin{problem} 

How many distinct strings can you make out of the characters
in the word ``potato-soup''?   (Include the hyphen ``\verb!-!'' as a character.)
\begin{explanation}
$$\answer{\sage{m(11,[2,3,2,1,1,1,1])} }$$
\end{explanation}

\end{problem}



\begin{problem} 
You need to form a battle group of 11 made up of orcs, elves, and
goblins.   In how many ways can you choose the composition of your battle group?
\begin{explanation}
$$\answer{\sage{binomial(13,11)}}$$
\end{explanation}
\end{problem}



\begin{problem} 
You need to form a battle group consisting of an orc, an elf, and
a goblin whose total strength is 12.   
The strength of each creature is 
an integer between 1 and 10 and the strength of the group is the sum of the
individual strengths.  In how many ways can you construct your battle group?
\begin{hint}
    This problem would be very much like the previous one if the minimum strength were not 1. How would you adjust it?
\end{hint}
\begin{explanation}
$$\answer{\sage{binomial(11,2)} }$$
\end{explanation}

\end{problem}



\begin{problem} 
You are out in the middle of the cornfields, where the (dirt) roads
all run north-to-south or east-to-west, evenly spaced at one road per mile.   
If you start
at a road intersection and wish to get to the intersection 5 miles to the
south and 7 miles west, how many different routes could you take?
\begin{hint}
Suppose you are writing a word using the letters ``s'' and ``w''. What is the length of this word? How often they must be repeated. Can you calculate all possible words if you know the length of the word and the frequencies of letters?
\end{hint}
\begin{explanation}
$$\answer{\sage{binomial(12,5)} }$$
\end{explanation}

\end{problem}



\begin{problem} 
Your latest cheap cell phone keyboard only includes the uppercase
alphabet (26 characters total).  How many 12-character strings can you
type that start with ``ST'' and contain no more than three T's?
\begin{explanation}
$$\answer{\sage{25^10+binomial(10,1)*25^9+binomial(10,2)*25^8}}$$
\end{explanation}

\end{problem}

\end{document}

