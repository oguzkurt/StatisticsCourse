\documentclass[handout]{ximera}
%\usepackage{hyperref}

%% handout
%% space
%% newpage
%% numbers 
%% instructornotes

%%% You can put user macros here
%% However, you cannot make new environments

\usepackage[letterpaper, total={6in, 8in}]{geometry}

\usepackage{tikz}
\usepackage{tkz-euclide}
\usetkzobj{all}

\tikzstyle geometryDiagrams=[ultra thick,color=blue!50!black]

%\usepackage{enumerate}
\usepackage{euler}

 
\usetikzlibrary{shapes,snakes}
\tikzset{
  dot hidden/.style={},
  line hidden/.style={},
  dot colour/.style={dot hidden/.append style={color=#1}},
  dot colour/.default=black,
  line colour/.style={line hidden/.append style={color=#1}},
  line colour/.default=black
}

%\begin{sagesilent}
def mydata(k):
    import numpy as np
    # Random test data
    np.random.seed(123)
    all_data = [np.random.normal(0, std, 100) for std in range(1, k)]
    return all_data

#boxplot
def seabox(my_data,mystr):
    import seaborn as sns
    sns.set_style("whitegrid")
    #tips = sns.load_dataset("tips")
    ax = sns.boxplot(x=my_data)
    ax.set(xlim=(0, None))    
    ax.get_figure().savefig(mystr,bbox_inches='tight')

#histogram
def myhist(k,b,m,M):
    from sage.plot.histogram import Histogram
    #k=data size
    # b= number of bins
    # random datA range
    a=histogram([randint(m,M) for _ in range(k)], bins=b)
    return a

#barchart & table
'''data1=[
['Zoo','giraffes', 'orangutans', 'monkeys'], ['SF Zoo', 20, 14, 23],['LA Zoo',12, 18, 29]
]'''
def mytb(data1,table_name,bar_name):
    import plotly.plotly as py
    import plotly.graph_objs as go
    from plotly.tools import FigureFactory as FF 
    from IPython.display import Image

    table1=FF.create_table(data1)
    py.iplot(table1,filename=table_name[:-4])
    py.image.save_as(table1, filename=table_name)
    Image(table_name)
    width=len(data1[0])
    height=len(data1)
    show(width)
    trace=[]
    d0=data1[0][1:width]
    show(d0)
    for i in range(height-1):
        show(i)
        d1=data1[i+1][1:width]
        show(d1)
        trace.append(
            go.Bar(
            x=d0,
            y=d1,
            name=data1[i+1][0]
                   )
                     )
    data = trace
    layout = go.Layout(
        barmode='group'
    )

    fig = go.Figure(data=data, layout=layout)
    py.iplot(fig, filename=bar_name[:-4])
    py.image.save_as(fig, filename=bar_name)
    Image(bar_name)
#piechart :
def mypie(labels,values,pie_name):
    import plotly.plotly as py
    import plotly.graph_objs as go
    from IPython.display import Image
    trace=go.Pie(labels=labels,values=values)
    py.iplot([trace], filename=data_name[:-4])
    py.image.save_as([trace],filename=pie_name)
    Image(pie_name)
    
\end{sagesilent}


 %% we can turn off include when making a master document
\usepackage[letterpaper, total={6in, 8in}]{geometry}
\outcome{Understand a second example of the Ximera style.}
\outcome{See how to include graphics.}

\title{Syllabus for Math 1312 \hfill
Spring 2017 \hfill \,}
\author{Oguz Kurt}
%\email{oguzkurt@gmail.com}
\date{Spring 2017}
\begin{document}
\begin{abstract}
Please, refer to this page for various information on this course.
\end{abstract} 
\maketitle

\section*{Course Information}
\begin{tabular}[c]{lll}
Class Times & : &  Monday - Wednesday 2:30 PM -- 3:45 PM  \\
Location & : &  824  \\
Instructor & : &  Oguz Kurt  \\
E-mail & : &   \href{mailto:oguzkurt@gmail.com}{oguzkurt@gmail.com} \\
Office & : & 831 \\
Office Hours & : & MW 11:00 AM -- 12:30 PM (and by appointment.)
\end{tabular}

\section*{Prerequisite(s)}

MATH 1311

\section*{Course Description}

This course is an overview of statistics. Topics included are statistics, simple random sampling, experimental designs, organizing data, descriptive measures, probability concepts, discrete and continuous random variables and normal distribution. We will also cover inferences and hypothesis tests for one and more population mean as well as sampling distribution of the sample mean.

\section*{Instructional Hours/Credits}

Lecture - 45 Clock Hours / 3 Semester Credits

\section*{Textbook}

We will be using a free, online textbook {\href{https://openstax.org/details/books/introductory-statistics}{\bf \Large Introductory Statistics}} prepared by the not-for-profit \href{https://www.openstax.org}{\bf OpenStax} collobaration within Rice University. You may support OpenStax by simply going to their webpage and donating as low as \$5.

\section*{Instructional Materials}

Handouts will frequently be provided. The students can use a calculator only for checking their work. You are allowed to use calculators.  The students might be asked to bring a notebook or tablet computer to class only to be used for group study purposes. Otherwise, the use of cell phones and, tablet or notebook computers is strictly prohibited. 

\section*{Instructional Methods}

Lectures, group study, online study materials.

\section*{Learning Outcomes}

Students will
\begin{itemize}
\item Develop a basic understanding on probability and statistics definitions.

\item Solve examples related to random variables and distributions.

\item Apply mean and variance in real-life cases.

\item Define and develop understanding on different discrete and continuous probability distributions, and distinguish the different distributions.

\item Explain random sampling and compute sample mean on different samples

\item Learn sample estimation, prediction intervals and construct knowledge on one sample and two sample problems

\item Test a statistical hypotheses, practice on different examples and real-life cases.

\item Investigate different tests and multiple comparisons, and interpret a case study for one-way experiment
\end{itemize}

\section*{Course Website(s)}

General course information, announcements, lecture materials will be delivered through the North American University Online system – NAU Moodle via \href{http://www.na.edu}{\bf http://www.na.edu}. The students can also monitor their grades. The login username and password are as same as the student computer account login information. The students are required to fill out their profile information in first login. Please visit IT Department if you have trouble in signing in.

The textbook for this course may be reached online at 
\\ 
\href{https://openstax.org/details/books/introductory-statistics}{\bf https://openstax.org/details/books/introductory-statistics}

The HW and some study materials will be provided online at 
\\ 
\href{http://ximera.osu.edu/course/oguzkurt/StatisticsCourse}{\bf http://ximera.osu.edu/course/oguzkurt/StatisticsCourse} 
\\
Note that you will need to sign-up to receieve HW credit. Please, use a gmail account that reflects your actual name so that I will not have to ask you which email is yours.

\section*{Homework and Expectations}

Students are expected to spend approximately six (6) hours a week, on average, completing homework assignments in order to achieve the learning objectives for this 15 week lecture course. This meets the Federal Government’s expectation of two hours of homework for each hour of lecture. 

Homework will be {\it regularly} assigned online via \href{http://ximera.osu.edu/course/oguzkurt/StatisticsCourse}{\bf our course page within The Ximera Project at The Ohio State University}. Students must also keep a small notebook for HW purposes only. They are expected to {\it tediously} write the full solution for each online HW problem in it. This notebook will be checked regulary to ensure that you are learning this course properly. 

\section*{Exams}

There will be 2 in class midterms that will cover the topics of this course incrementally. Tests are closed book. Calculators are allowed. See the course outline at the end of this document for tentative exam dates. No make-up exams will be given unless a very clear, official documentation of your absence is provided.

There will also be a {\bf cumulative} final exam at the end of the semester.

\section*{Academic Honesty}

Each student assumes the responsibilities of being a member of the NAU academic community.  All acts of plagiarism are not tolerated including: cheating, claiming one’s work as their own, fabrication and helping one to commit any of these acts.  Any violations of academic honesty will receive strict disciplinary action, which can include suspension and even expulsion from NAU.  

\section*{Accommodatations}

Students that require any accommodation (such are students with disabilities, religious conflicts, etc…) should notify the instructor as early as possible and accommodations will be made on an individual basis in adherence with the regulations outlined in the Student Handbook.

\section*{Assessment Criteria \& Methods of Evaluating Students}

\begin{tabular}[c]{lllcccl}
2 Midterms  & : & 40\% & \,\hspace{2cm} &  & & \\  
Final & : & 30\% & \hspace{2cm} & & & \\  
Homework & : & 15\% & \hspace{2cm} & & & \\  
Quiz & : & 15\% & \hspace{2cm} & &\\  
Attendance & : & 5\% & \hspace{2cm} & &  & \\  
\end{tabular}

Anyone who receives 92\% or above from the final exam receives an A as their letter grade for this course.

\section*{Course Outline}

To be added online

%\begin{instructorNotes}
%  Here we see a multi-part question.
%\end{instructorNotes}

%\begin{instructorIntro}
%  This should tell me something
%\end{instructorIntro}


\end{document}

